\documentclass{zjut-report}

\usepackage[utf8]{inputenc}
\usepackage[colorlinks,linkcolor=black]{hyperref}
\usepackage{pdfpages}
\usepackage{booktabs}
\usepackage{multirow}
\usepackage{longtable}
\usepackage{multicol}

\usepackage{tikz}  
\usetikzlibrary{arrows,shapes,chains}  
\tikzstyle{startstop} = [rectangle, rounded corners, minimum width=3cm, minimum height=1cm, text centered, draw=black]
\tikzstyle{io} = [trapezium, trapezium left angle = 70, trapezium right angle=110, minimum width=3cm, minimum height=1cm, text centered, draw=black]
\tikzstyle{process} = [rectangle, minimum width=3cm, minimum height=1cm, text centered, draw=black]
\tikzstyle{decision} = [diamond, minimum width=3cm, minimum height=1cm, text centered, draw=black, aspect=2, inner sep=2pt, node distance=5mm, thick]
\tikzstyle{arrow} = [thick,->,>=stealth]
\tikzstyle{box} = [rectangle, rounded corners, minimum width=1.5cm, minimum height=1cm, text centered, draw=blue!40, fill=blue!8]
\tikzstyle{tree} = [circle, minimum width=1cm, minimum height=1cm, text centered, draw=blue!40, fill=blue!8]
\tikzstyle{tree2} = [circle, minimum width=1cm, minimum height=1cm, text centered, draw=green!40, fill=green!8]
\tikzset{global scale/.style={
    scale=#1,
    every node/.append style={scale=#1}
  }
}
\usepackage{pgf-umlcd}

\definecolor{dkgreen}{rgb}{0,0,0}
\definecolor{gray}{rgb}{0.5,0.5,0.5}
\definecolor{mauve}{rgb}{0.58,0,0.82}

\lstset{frame=tb,
  language=c++,
  aboveskip=3mm,
  belowskip=3mm,
  showstringspaces=false,
  columns=flexible,
  basicstyle={\small\ttfamily},
  numbers=left,
  numberstyle=\tiny\color{gray},
  keywordstyle=\color{blue},
  commentstyle=\color{dkgreen},
  stringstyle=\color{mauve},
  breaklines=true,
  breakatwhitespace=true
  tabsize=2
}

\lhead{数据库系统课程设计报告}
\rhead{\leftmark}

\begin{document}

\includepdf[pages={1}]{cover.pdf}

% ================== 封面部分 ==================
\begin{titlepage}
\begin{center}
% 2号黑体
{\fontsize{36pt}{43pt}\selectfont\heiti XXX系统设计与实现}

\vspace{2cm}

% 3号仿宋_GB2312
{\fontsize{16pt}{19pt}\selectfont\CJKfontspec{FangSong_GB2312} 作者名(小组组长)$^1$, 作者名2$^2$, 作者名3$^3$, 作者名4$^4$}

\vspace{1cm}

% 6号宋体
{\fontsize{8pt}{10pt}\selectfont\songti
$^1$(浙江工业大学计算机学院XX专业)\\
$^2$(浙江工业大学计算机学院XX专业)\\
$^3$(浙江工业大学计算机学院XX专业)\\
$^4$(浙江工业大学计算机学院XX专业)
}

\vfill
{\large \today}
\end{center}
\end{titlepage}

% ================== 摘要和关键词 ==================
\newpage
\begin{center}
% 摘要标题(黑体小四)
{\fontsize{12pt}{15pt}\selectfont\heiti 摘\quad 要}
\end{center}

\vspace{1cm}

% 摘要内容(小5号宋体)
{\fontsize{9pt}{11pt}\selectfont\songti
中文摘要内容置于此处,字体为小5号宋体。
}

\vspace{1cm}

% 关键词(小5号宋体)
{\fontsize{9pt}{11pt}\selectfont\songti
\textbf{关键词:}关键词;关键词;关键词;关键词
}

% ================== 目录 ==================
\tableofcontents
\newpage

% ================== 正文结构小标题 ==================
% 一级标题:4号黑体
\chapter{1\quad 需求分析}

% 二级标题:5号黑体
\section{1.1\quad 数据需求描述}
\subsection{1.1.1\quad 三级标题}

\section{1.2\quad 系统功能需求}
\section{1.3\quad 其他性能需求}

\chapter{2\quad 概念结构设计}
\section{2.1\quad 局部E-R图}
\section{2.2\quad 全局E-R图}
\section{2.3\quad 优化E-R图}

\chapter{3\quad 逻辑结构设计}
\section{3.1\quad 关系模式设计}
\section{3.2\quad 数据类型定义}
\section{3.3\quad 关系模式的优化}
\section{3.4\quad 关系模式的优化}

\chapter{4\quad 物理结构设计}
\section{4.1\quad 聚簇设计}
\section{4.2\quad 索引设计}
\section{4.3\quad 分区设计}

\chapter{5\quad 数据库实施}
\section{5.1\quad 基本表建立}
\section{5.2\quad 视图的建立}
\section{5.3\quad 索引的建立}
\section{5.4\quad 触发器建立}
\section{5.5\quad 建存储过程}

\chapter{6\quad XXXX系统开发与运行}
\section{6.1\quad 开发平台和开发环境介绍}
\section{6.2\quad 数据准备}
\section{6.3\quad 应用系统的开发}

本系统采用成熟的Web开发技术栈:
\begin{itemize}
\item 前端:HTML、CSS、JavaScript、JSP
\item 后端:Java Servlet、JavaBean
\item 数据库:MySQL 8.0
\item 服务器:Tomcat 9.0
\end{itemize}

这些技术都是经过实践验证的成熟技术,具有丰富的开发资源和社区支持,技术风险较低。

\subsubsection{经济可行性}

系统开发成本主要包括:
\begin{itemize}
\item 开发人员成本
\item 服务器和数据库软件成本
\item 维护和升级成本
\end{itemize}

相比传统管理方式,系统投入使用后能够显著降低人工成本,提高工作效率,具有良好的经济效益。

\subsubsection{操作可行性}

系统采用直观的Web界面,操作简单易学,用户培训成本低。系统支持多角色权限管理,能够满足不同用户的使用需求。

\subsection{应用环境}

\subsubsection{硬件环境}
\begin{itemize}
\item 服务器:支持Java运行环境的服务器
\item 客户端:支持现代浏览器的计算机设备
\item 网络:稳定的网络连接
\end{itemize}

\subsubsection{软件环境}
\begin{itemize}
\item 操作系统:Windows/Linux/macOS
\item 数据库:MySQL 8.0及以上版本
\item Web服务器:Apache Tomcat 9.0及以上版本
\item 浏览器:Chrome、Firefox、Safari、Edge等现代浏览器
\end{itemize}

\section{数据需求描述}

\subsection{数据字典}

系统涉及的主要数据实体及其属性如下:

\subsubsection{用户信息表(User)}
\begin{itemize}
\item 用户ID (user\_id):主键,唯一标识用户
\item 用户名 (username):用户登录名,唯一
\item 密码 (password):用户登录密码,加密存储
\item 用户类型 (user\_type):用户角色(管理员/教师/学生)
\item 创建时间 (create\_time):用户创建时间
\end{itemize}

\subsubsection{学生信息表(Student)}
\begin{itemize}
\item 学生ID (student\_id):主键,学号
\item 姓名 (name):学生姓名
\item 性别 (gender):学生性别
\item 出生日期 (birth\_date):学生出生日期
\item 专业 (major):学生所属专业
\item 班级 (class):学生所在班级
\item 入学年份 (enrollment\_year):学生入学年份
\end{itemize}

\subsubsection{教师信息表(Teacher)}
\begin{itemize}
\item 教师ID (teacher\_id):主键,工号
\item 姓名 (name):教师姓名
\item 性别 (gender):教师性别
\item 职称 (title):教师职称
\item 所属院系 (department):教师所属院系
\item 联系电话 (phone):教师联系电话
\end{itemize}

\subsubsection{课程信息表(Course)}
\begin{itemize}
\item 课程ID (course\_id):主键,课程编号
\item 课程名称 (course\_name):课程名称
\item 学分 (credits):课程学分
\item 授课教师 (teacher\_id):外键,关联教师表
\item 课程类型 (course\_type):课程类型(必修/选修)
\item 开课学期 (semester):开课学期
\end{itemize}

\subsubsection{选课信息表(Enrollment)}
\begin{itemize}
\item 选课ID (enrollment\_id):主键
\item 学生ID (student\_id):外键,关联学生表
\item 课程ID (course\_id):外键,关联课程表
\item 选课时间 (enrollment\_time):选课时间
\item 成绩 (score):学生成绩
\end{itemize}

\subsection{数据流图}

\subsubsection{初级数据流图}

系统的主要数据流包括:
\begin{enumerate}
\item 用户登录认证流程
\item 学生信息管理流程
\item 课程信息管理流程
\item 成绩录入与查询流程
\item 成绩统计分析流程
\end{enumerate}

\subsubsection{详细级数据流图}

\paragraph{用户登录认证流程}
\begin{enumerate}
\item 用户输入用户名和密码
\item 系统验证用户身份
\item 根据用户类型跳转到相应界面
\item 记录登录日志
\end{enumerate}

\paragraph{成绩管理流程}
\begin{enumerate}
\item 教师登录系统
\item 选择要录入成绩的课程
\item 查看选课学生列表
\item 录入学生成绩
\item 系统保存成绩数据
\item 生成成绩统计报告
\end{enumerate}

\subsection{系统开发边界}

系统的主要功能边界包括:

\begin{itemize}
\item \textbf{用户管理}:用户注册、登录、权限管理
\item \textbf{学生管理}:学生信息的增删改查
\item \textbf{教师管理}:教师信息的增删改查
\item \textbf{课程管理}:课程信息的增删改查
\item \textbf{选课管理}:学生选课、退课操作
\item \textbf{成绩管理}:成绩录入、修改、查询
\item \textbf{统计分析}:成绩统计、排名分析
\item \textbf{系统维护}:数据备份、系统配置
\end{itemize}

系统不包括的功能:
\begin{itemize}
\item 在线考试功能
\item 教学资源管理
\item 财务收费管理
\item 宿舍管理
\item 图书馆管理
\end{itemize}

\end{document} 